%%
% 致谢
% 谢辞应以简短的文字对课题研究与论文撰写过程中曾直接给予帮助的人员(例如指导教师、答疑教师及其他人员)表示对自己的谢意,这不仅是一种礼貌,也是对他人劳动的尊重,是治学者应当遵循的学术规范。内容限一页。
% modifier: 黄俊杰
% update date: 2017-04-15
%%

\chapter{致谢}
首先,对于指导老师,我要由衷地表达对他耐心的指导和细心的检查表示感谢。老师对论文写作结构和实验思路等方面造诣颇深,给了我很多关键的指导,让我受益匪浅。

这是我的学位毕业论文,也是第一篇论文,我不能因为毕业就心甘止步于此,而是要继续好好学习,学习指导老师的学者素养和求知精神,在软件工程领域开拓进取,把理论投于实践,努力成为有学问有能力的社会需要的人才,不负指导老师和母校中山大学各位老师的谆谆教诲。

其次,感谢我们的学院老师们和给我帮助的所有同学,老师们为我解答了学术和生活上许许多多的疑惑,还有很多同学们也分享了他们的论文心得和写作技巧,让我认识到同学之间需要互帮互助、共同奋斗,如此的温暖和团结的氛围让我庆幸能成为其中的一员,我将一生铭记这段难忘的时光。

最后感谢我的家人,是他们多年来对我学业和生活的坚定支持,我才能如此幸运地在中山大学踏踏实实的读书,使我得以顺利完成本科学业。

\vskip 108pt
\begin{flushright}
	陈环\makebox[1cm]{} \\
	\today
\end{flushright}

